\section{Workshop Format}
\label{sec:Format}

\begin{description}
   \item[Deadlines \& Paper Format] Cf. CfP in Appendix
   
   \item[Evaluation Process] At least three reviewers will evaluate each submission.
   Full research papers will be reviewed using standard scientific criteria: alignment 
   with the workshop topic(s), novelty, evaluation, and ability to generate discussion.
   Short papers will be evaluated based on their likelihood to spark lively 
   discussions.

   \item[Intended Workshop Format] In the ideal case of a full day workshop, we 
   will host two keynotes, ideally one from academia and the other from industry
   (depending on how people are available). 
   A morning keynote will set the stage for the rest of the day, followed by 
   paper presentations. A second keynote after the lunch break will set the 
   discussion around exemplar presentations and new and provocative ideas,
   which will foster discussions and out-of-the-box ideas. Each talk throughout the
   day will be followed by discussions.
   The afternoon will reserve time to discuss the examplars, by starting discussions
   around identifiable patterns, common practices, popular formalisms/languages/tools, etc. 
   The workshop will end with a wrap-up discussion to formulate the workshop's 
   conclusion, identify open challenges, and outline future work.
   A summarizing publication will be included in the proceedings. 
   Depending on the quality of papers/discussions, we may organise a Special Issue
   based on invitations (this/previous edition(s)) and open call.
   
   \item[(Expected) Participants Number] 25-35 (similar to previous editions.)
   
   \item[Equipment] Whiteboard + Slide Projection
\end{description}

%\subsection{Deadlines \& Paper Format: cf. CfP}
%\label{sec:Deadlines-Format}
%
%\subsection{Evaluation Process}
%\label{sec:EvaluationProcess}
%
%At least three reviewers will evaluate each submission.
%Full research papers will be reviewed using standard scientific criteria: alignment 
%with the workshop topic(s), novelty, evaluation, and ability to generate discussion.
%Short papers will be evaluated based on their likelihood to spark lively 
%discussions at the workshop.
%
%\subsection{Intended Workshop Format}
%\label{sec:Format-Workshop}
%
%In the ideal case of a full day workshop, we will host two keynotes, ideally 
%one from academia and the other from industry (depending on how people are 
%available). A morning keynote will set the stage for the rest of the day, 
%followed by paper presentations. A second keynote after the lunch break
%will set the discussion around exemplar presentations and new and provocative ideas,
%which will foster discussions and out-of-the-box ideas. Each talk throughout the
%day will be followed by discussions.
%
%The afternoon will reserve time to discuss the examplars, by starting discussions
%around identifiable patterns, common practices, popular formalisms/languages/tools, etc. 
%The workshop will end with a wrap-up discussion to formulate the workshop's 
%conclusion, identify open challenges, and outline future work.
%A summarizing publication will be included in the proceedings.
%
%\subsection{(Expected) Participants Number: 25-35 (similar to previous editions.)}
%
%\subsection{Equipment: Whiteboard}



