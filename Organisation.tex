\section{Organisation Details}
\label{sec:Organisation}

% RP: TO BE DISCUSSED...

% We are joined by two new organising members, Joeri Exelmans and Randy Paredis, 
% who will take care of the Publicity and Website, paired with more senior organisers.

The organisation team is made up of a good mix of junior and senior organisers.

A \textbf{tentative} PC is given in the CfP in Appendix.
We request that MPM4CPS be run on its own as we plan half a day of discussions.
We can merge with another workshop if this is a condition from the organization 
and the topics are similar enough.

\medskip
\noindent
\textbf{Moussa Amrani} obtained his Ph.D. in 2013 from the University of Luxembourg. 
He is currently a postdoctoral researcher at the University of Namur and the Namur Digital Institute in Belgium. 
He is (co-)author of over 50 papers on MDE, IoT and formal verification published in international conferences (MODELS, SLE, ECMFA, ASE, ETAPS, NFM, CAiSE, \ldots) and journals (SoSyM, JSS, TSE, ToSEM, JoT, IST, ComLan, \ldots). He co-founded and co-organized the VoLT workshop at MoDELS from its inception in 2013, and co-organised
the previous editions of the MPM4CPS Workshop.

\medskip
\noindent
\textbf{Dominique Blouin} is a research engineer at Telecom Paris, Institut Polytechnique Paris 
(France). He obtained an M.Sc. in Physics (Canada) and a Ph.D. in Computer Science (France) in 2013. He worked for many years in industry as a software architect and was the vice-chair of the Foundational Aspects Working Group in the MPM4CPS COST Action. He has been an active member of the SAE AADL standardization committee for the past 10 years. His research interests are MPM, model management, model transformation and (bi-directional) synchronization, requirements engineering, CPS.

\medskip
\noindent
\textbf{Moharram Challenger} is a research professor at the University of Antwerp, Belgium. He was the CTO of a software company involved-in/leading several national and international software intensive projects. He has served as an organisation committee member for SummerSim'20, ICSMM'20, AnnSim'21, IWCPS@FedCSIS'21, and AMSC'21. Also, he played the role of co-chair for several workshops organised in MoDELS and STAF 2020-21 (MDE-Intelligence, MPM4CPS, MDE4IoT, SERP4IoT, SEDES, MESS, EMAS, etc.).

\medskip
\noindent
\textbf{Joeri Exelmans} is a Ph.D. student at the University of Antwerp, Belgium, and has worked on a Flanders Make project aiming to facilitate collaboration in complex engineering workflows. His research interests are the engineering of hybrid modeling languages, model versioning, and inconsistency management and traceability in complex engineering workflows.

\medskip
\noindent
\textbf{Randy Paredis} is a Ph.D. student at the University of Antwerp, Belgium, and is exploring architectures and frameworks for model-based design of Digital Twins within the context of Industry 4.0. His research interests are multi-paradigm modelling, using DEVS as a common denominator for discrete-event modelling languages, and co-simulation.


\medskip
\noindent
\textbf{Robert Heinrich} holds a Ph.D. from Heidelberg University, and is head of the Quality-driven System Evolution research group at Karlsruhe Institute of Technology (KIT). His research interests include modularization
and composition of model-based analysis for performance, confidentiality and maintainability, etc. applied to information systems, business processes and automated production systems. One core asset of his work is the Palladio software architecture simulator. He is involved in the organization committees of several international conferences, established and organized various workshops, is reviewer for international premium journals (IEEE TSE and IEEE Software), and academic funding agencies.

