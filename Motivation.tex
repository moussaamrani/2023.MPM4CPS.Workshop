\section{Motivation}
\label{sec:Motivation}

%\subsection{Objectives and Scope}
%\label{sec:Objectives-Scope}

\noindent
\textbf{Objectives and Scope.}
%
Tackling the complexity involved in developing truly complex, designed systems 
is a topic of intense research and development.
In the past, system complexity has drastically increased once software components 
were introduced in the form of embedded systems, controlling physical parts of 
the system, and has only grown in CPS, where the networking aspect of the systems 
and their environment are also taken into account.
The complexity faced when engineering CPS is mostly due to the plethora of 
cross-disciplinary design alternatives and inter-domain interactions.
To date, no unifying theory nor system design methods, techniques, or tools to 
design, analyze, and ultimately deploy CPS exist.
Individual (physical systems, network, software) engineering disciplines offer 
only partial solutions and are no match for CPS complexity.



\emph{Multi-Paradigm Modeling (MPM)} offers a foundational framework for gluing the 
several disciplines together in a consistent way.
The inherent complexity of CPS is broken down into different levels of 
abstraction and views, each expressed in appropriate modeling formalisms.
MPM offers processes and tools that can integrate the views, abstractions and 
components that make up a CPS.

MPM encompasses many research topics: from language engineering (for DSLs, 
including their (visual/textual) syntax and semantics), to processes to support multi-view 
and multi-abstraction modelling, simulation for full-system analysis, and deployment.
The added complexity that CPS bring compared to embedded and software-intensive 
systems requires consideration of how MPM techniques can be applied or adapted 
to these new applications, tying together multiple domains.
Many remaining research questions require answers from researchers in different 
domains, as well as a unified effort from researchers that work on supporting 
techniques and technologies.
The community needs a workshop setting to meet up and align past and future 
research activities.


\smallskip\noindent
\textbf{Workshop's Purpose.} During this Workshop, we want to bring together 
researchers and practitioners in the area of MPM (specifically applied to 
developing CPS) in order to identify possible points of synergy, common problems
and solutions, tool building aspects and the vision for the future of the area.
The goal is to organize a \textbf{highly interactive workshop}, with a significant 
portion of the Workshop dedicated to \textbf{discussions}.
\textbf{"Regular" Research papers} from academic and industry authors 
   will present novel research results on the Workshop's topics of interest. 
   We will encourage the submission of out-of-the-box presentations, which are 
   not deeply researched yet, but can lead to new insights, discussions, and 
   future collaborations.

Similar to last year, we will invite the submission of 
   \textbf{Exemplars}, i.e., typical, yet relatively tractable use cases of CPS 
   demonstrating typical activities required for CPS Engineering, and explicitly
   detailing the underlying formalisms, languages and tools deployed to support 
   such activities, all expressed in a similar way to enable comparison and 
   extract CPS Engineering common practices and design patterns.

% Those directions were not possible during the two previous editions due to the
% virtual nature of the Workshop, but we hope to foster new topics and have small
% groups discussing and brainstorming around the exemplars we collected so far 
% (including the ones resulting from the previous edition).


%\subsection{Intended Audience}
%\label{sec:Audience}

\smallskip
\noindent
\textbf{Intended Audience.}
%
The intended audience includes researchers as well as practitioners who are 
interested in MPM techniques in the context of CPS development.
We expect to attract many attendees of earlier MPM-related events, those who 
contributed to the COST action, as well as a broader audience.
This includes researchers that work on the fundamentals of language 
engineering, (visual) modelling environment construction, (co-)simulation 
techniques, as well as tool builders and users of these tools.


%\subsection{Topics of Interest}
%\label{sec:ToI}

\smallskip
\noindent
\textbf{Topics of Interest.}
%
A list of topics of interest is given in the CfP available in Appendix.
Note that we have explicitly included \emph{classification} and \emph{exemplar} 
topics, as it is key to structure and discuss the future of MPM. %, and integrated 
% broad topics related to \#MDE4SG (in the two last points).

%\begin{itemize}
   %\item Foundations of domain-specific modelling, with a particular focus on 
   %\emph{classifications} of the various dimensions around MPM (formalisms; processes;
   %related activities such as V\&V, deployment, calibration, etc.; tools, and 
   %methodologies);
   %\item Modelling language engineering, modular design of modelling languages, 
   %with a particular focus on de-/composition;
   %\item Co-simulation, coordination algorithms ensuring correct simulation results.
   %\item Digital twins of complex systems and their relationship to MPM techniques.
   %\item Applications of MPM techniques in automotive, aviation, manufacturing, etc.
   %\item MPM for (self-)adaptive systems
   %\item Machine Learning applied in an MPM context, Smart CPS
   %\item Social impacts processes in CPS, Large Data Management Modelling in CPS
%\end{itemize}

%\subsection{Relevance}
%\label{sec:Relevance}

\noindent
\textbf{Relevance.}
%
The MODELS conference is an ideal venue for organizing MPM4CPS since it brings 
together researchers that aim to advance the state-of-the-art in model-driven 
engineering and practitioners who have valuable application experiences to share.

Furthermore, in the previous edition at MODELS, we noticed a large interest from the modeling community (with about 60 participants in the morning session). This is a clear indicator that the workshop is actively relevant in the community.



%\subsection{Context}
%\label{sec:Context}

\smallskip
\noindent
\textbf{Context.}
%
The MPM community has been actively researching new techniques for system design 
for over a decade, through many related events.
One-week Computer Automated Multi-Paradigm Modeling (CAMPaM) workshops have been 
organized yearly since 2004 at McGill University’s Bellairs campus, Barbados. 
Additionally, the International Summer School on Domain-Specific Languages - 
Theory and Practice (DSM-TP), focused on the education of language engineering 
techniques, and has been organized since 2009.
Its target audience includes Ph.D. students, researchers, and software industry 
professionals.
Most recently, a European Cooperation in Science and Technology (COST) research 
network has been active since 2015 on the use of MPM techniques for designing 
CPS, bringing together 29 European partner countries\footnote{https://www.cost.eu/actions/IC1404/ 
and http://mpm4cps.eu/}.
The chair and co-chair of this network (Hans Vangheluwe and Vasco Amaral) are 
members of the steering committee for this workshop.

The first edition of the workshop has lead to the preparation of a theme section
on the topic of MPM4CPS for the SoSyM journal (which has been finally published%
\footnote{ https://link.springer.com/article/10.1007/s10270-021-00882-1}).
The second and third editions were operated fully online, and therefore have been
reduced to half-day workshops consisting solely of paper presentations, albeit with
longer discussion periods (typically, 10-15 minutes after each presentation), and
each paper presentation session still attracted ca. 20-25 participants with lively,
interesting discussions. The fourth edition provided a hybrid virtual/in-person solution at the MODELS 2022 conference. The morning session attracted about 60 people (in-person), with many still actively partaking in the lively discussions at the end. We suspect there will be at least the same amount of interest this year.
% Because of the virtual/online nature of these editions,
% hosting an interactive/''brainstorming'' session was difficult, but we strongly
% believe those sessions are at the heart of MPM4CPS, and will attract enough
% participants (around 10, not counting the organisation members).

All these initiatives, as well as the success during the online editions, 
demonstrate both the continued relevance of the topic, and the 
potential impact of a new edition of the workshop.



%\subsection{Needs}
%\label{sec:Needs}

\smallskip
\noindent
\textbf{Needs.}
%
The MPM4CPS workshop (series) follows the successful series of nine MPM workshops that 
were organized as part of the MODELS conference during the years 2006 through 2015.
These workshops attracted many participants in the past, and we saw that the topic
is still relevant, since the previous editions attracted many submissions and lively discussions.
These two reasons make this workshop worth organizing at MODELS this year.

% RP: I was unable to find any information about this for 2023, so I commented this section.
% Historically, GeMoC and EXE, now merged into the MLE (Modeling Language 
% Engineering and Execution), were the closest workshops planned to be organised 
% in MoDELS this year. They focus mainly on two topics: the globalization of
% modeling techniques, which include techniques and processes to create and
% integrate heterogeneous languages, and the execution, animation and debugging
% of modeling languages. While both are concerned with specific aspects of 
% \emph{software} language engineering, MPM includes modeling languages for 
% physical domains (e.g., electrical, mechanical, etc.) that require 
% continuous-time solvers for simulation, and clean integration with other (SW/HW) 
% languages.
% MPM4CPS further differentiates from those workshops by focusing on CPS. 
% We believe that both workshops can strengthen each other by focusing on different 
% (specialized) aspects of challenges within the MODELS community.

The challenges to design and develop CPS, with a focus on MPM techniques as a 
foundational framework for supporting the multi-domain models, tools, and 
processes are fundamental enough to warrant a focused workshop.
